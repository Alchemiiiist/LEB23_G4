\chapter{Linux基础}
\section{Linux及操作系统简介}

Linux是一种免费的开源操作系统。它基于Unix操作系统,由Linus Torvalds于1991年创建。Linux用于许多不同的设备,例如服务器,超级计算机,智能手机甚至汽车。它以其稳定性,安全性和灵活性而闻名。Linux有许多不同的发行版或“distros”,例如Ubuntu,Fedora,CentOS,Gentoo,Arch Linux等。

Linux是基于命令行的操作系统,这意味着用户使用文本命令与之交互,而不是像Windows或Mac OS X那样使用图形用户界面(GUI)。它使用户对其系统拥有更多的控制权,并使他们更容易自动化任务。

\subsection{Linux发行版}
Linux发行版是一个由Linux内核、GNU工具、附加软件和软件包管理器组成的操作系统。它也可能包括显示服务器和桌面环境,以用作常规的桌面操作系统。这个术语之所以是“Linux发行版”,是因为像Debian、Ubuntu这样的机构“发行”了Linux内核以及所有必要的软件及实用程序(如网络管理器、软件包管理器、桌面环境等),使其可以作为一个操作系统使用。

一些流行的Linux发行版包括Ubuntu,Fedora,CentOS,Gentoo,Arch Linux等。这些发行版都有自己的特点和优点,列举如下:
\begin{itemize}
    \item Fedora:面向开发人员,新版本的软件包更新频繁。
    \item Debian:稳定,易于使用,适合服务器。
    \item Ubuntu:易于使用,适合桌面用户。
    \item CentOS:企业级操作系统,稳定性高。
    \item Gentoo:高度可定制,适合高级用户。
    \item Arch Linux:轻量级和自定义。
\end{itemize}

在本学期中,我们使用的是Ubuntu。

\section{Linux教程网站}

\begin{enumerate}
    \item Data Science at the Command Line
          https://datascienceatthecommandline.com/2e/
    \item 菜鸟
          https://www.runoob.com/linux/linux-tutorial.html

    \item 自由的GNU/Linux发行版
          https://www.gnu.org/distros/free-distros.html
    \item Ubuntu问答社区
          https://askubuntu.com/
    \item 软件开发者
          https://www.csdn.net/
\end{enumerate}


\section{Linux基本命令}

\subsection{服务器有关命令}

\subsubsection{登陆服务器}
windows系统: window  powershell

mac 系统: 终端(我的系统)

使用ssh指令来进行用户名的登陆,如: ssh leb4b@117.78.18.116。登陆后,需要输入密码

注:输入密码时,密码会进行保密,因此在界面上看不到输入的密码


\subsubsection{查看服务器状态}

\begin{enumerate}
    \item 查看服务器此时登陆用户名名单

          指令1:
          \begin{lstlisting}
w | less
\end{lstlisting}

          指令2:
          \begin{lstlisting}
who | less
\end{lstlisting}

    \item 查看我的用户名

          指令:
          \begin{lstlisting}
whoami
\end{lslisting}
\item 修改自己用户的密码
\begin{lstlisting}
指令:   passwd
\end{lstlisting}
    \item 显示系统运行状态

          指令:
          \begin{lstlisting}
top
\end{lstlisting}

\end{enumerate}

\subsection{文件操作指令}
\subsubsection{list指令}

\begin{enumerate}
    \item 查看下属文件
          \begin{lstlisting}
ls
\end{lstlisting}

    \item 查看下属文件时,同时查看文件的属性信息
          \begin{lstlisting}
ls -l
\end{lstlisting}

    \item 按照文件名大小排序
          \begin{lstlisting}
ls -s
\end{lstlisting}

    \item 按照时间排序
          \begin{lstlisting}
ls -t 
\end{lstlisting}

    \item 按照字节大小排序
          \begin{lstlisting}
ls -h 
\end{lstlisting}

    \item 显示目录下以.FASTA结尾的文件
          \begin{lstlisting}
ls *. FASTA 
\end{lstlisting}

    \item 显示目录下所有文件,并且一行只显示一个
          \begin{lstlisting}
ls -1 
\end{lstlisting}

    \item 列出根目录
          \begin{lstlisting}
ls ~
\end{lstlisting}

    \item 逐级显示当前目录下所有子目录和文件
          \begin{lstlisting}
ls -lR
\end{lstlisting}

    \item 逐屏显示当前目录下所有子目录和文件详细信息
          \begin{lstlisting}
ls -lR | less 
\end{lstlisting}

\end{enumerate}

\subsubsection{cd指令}
\begin{enumerate}

    \item 进入ABC子目录
          \begin{lstlisting}
cd ABC
\end{lstlisting}

    \item 返回根目录
          \begin{lstlisting}
cd 
\end{lstlisting}

    \item 进入子目录ABC下面的二级子目录EFG
          \begin{lstlisting}
cd ABC/EFG
\end{lstlisting}

    \item 返回上级目录
          \begin{lstlisting}
cd ..
\end{lstlisting}
\end{enumerate}


\subsubsection{新建、复制、删除、移动文件操作}
\begin{enumerate}
    \item 新建文件

          \begin{lstlisting}
mkdir                #建立文件名
mkdir 0306 0313 0320 #建立以0306 0313 0320命名的多个文件夹
mkdir 0227/HB        #在子目录0227下建立二级子目录HB
\end{lstlisting}

          \begin{lstlisting}
cat > 文件夹名
control c   #并且可以输入信息,退出该文件指令为:
cat 文件名   #此外还可以通过查看指令来看文件内的信息
\end{lstlisting}

          \begin{lstlisting}
nano 文件名
\end{lstlisting}

    \item 复制文件

          \begin{lstlisting}
cp a b   #将a复制,命名为b
\end{lstlisting}

    \item 重命名文件名
          \begin{lstlisting}
mv a b   #将a的名字改成b
\end{lstlisting}

    \item 删除文件
          \begin{lstlisting}
rm b   #将b文件夹删掉
       #注意:chmod -w 文件 ,该指令能够改变文件属性,使之不能够被删除
\end{lstlisting}

    \item 建立软连接

          \begin{lstlisting}
ln -s #源文件 目标文件
\end{lstlisting}
\end{enumerate}

\subsubsection{其他常用的指令}
\begin{enumerate}
    \item
          查找文件的位置(例如查找needle的位置)
          \begin{lstlisting}
which needle
whereis needle
local needle
\end{lstlisting}


    \item 显示系统存储空间不同分区名称、容量和使用状态
          \begin{lstlisting}
df
\end{lstlisting}


    \item 显示当前目录下所有子目录和文件占用空间
          \begin{lstlisting}
du
\end{lstlisting}

    \item 按MB为单位显示/tmp目录占用空间
          \begin{lstlisting}
du –m /tmp
\end{lstlisting}

    \item
          将文件209HBA.FASTA压缩
          \begin{lstlisting}
gzip 209HBA.FASTA
\end{lstlisting}

    \item 将子目录0307中及其子目录中所有文件生成归档文件
          \begin{lstlisting}
tar –cvf 0307.tar 0307
\end{lstlisting}

    \item 输入路径自动补齐文件名:tap
    \item 重复上一次使用的指令:上箭头


\end{enumerate}