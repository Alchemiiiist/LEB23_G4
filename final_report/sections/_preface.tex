\chapter*{前言}

这本册子是2023年春季学期《Linux生物信息技术基础》课程的期末总结。Linux生物信息技术基础(Linux-based Essential Bioinformatics, LEB)是由罗静初教授\footnote{罗静初,北京大学生命科学学院教授,博士生导师,欧洲分子生物学网络组织中国节点负责人,英国Briefings in Bioinformatics杂志编委。1947年生,1970年毕业于北京大学生物系。1986年起从事DNA和蛋白质序列计算机分析。1987-1989年赴美国马里兰大学进修访问,从事蛋白质分子模型和计算机在分子生物学中的应用研究。1991-1999年先后5次赴英国帝国癌症研究所合作研究,从事蛋白质分子模型、蛋白质结构域分析和数据库构建、蛋白质回环数据库构建等研究。1996年起主持和参加863、973、211、985,以及自然科学基金委、教育部、北京市科委等项目。 发表学术论文30多篇,主持翻译《生物信息学概论》、合作编写《生物信息学》、《分子生物学前沿技术》,开设“实用生物信息技术”研究生课程。}开设的课程,罗老师以生物信息为主题,以Linux系统为载体,以上课和小组讨论为形式,带领初学者步入生物信息的大门,让有基础的学习者也能有所收获。

万分有幸的是,我们处在一个和谐快乐的小组,特色鲜明的各位组员让我们能各取所长,共同进步。高大可正从事蛋白质翻译调控相关研究;邓昆月;作为本科生,唐明川在空间转录组方向进行科研实践,打磨科研技能,发展科学思维,同时正在积极探索开拓自己的能力边界;吴航锐。

学期末,虽无考试压力,但一同回顾半年所得,梳理知识技能,也是一件令人愉悦的事情;为了使报告不只是考核,我们力图增加它的交流性,因而你可以看到每位成员都留下了自己的照片和联系方式,期待能够和大家建立联系,以我们所学帮助到大家。从课程整体来看,课程内容比较灵活,与生物信息相关的主线在于蛋白质和核酸序列处理,包括序列比对和ChIP-seq、RNA-seq等过程的上下游分析。主线之外,根据同学需要,老师安排了GitHub,TBtools使用介绍和基于conda、vsCode、zshell等的环境配置,这让我们受益匪浅。

我们选择从\textbf{熟悉Linux系统}到\textbf{各个主题应用}的逻辑将所学内容分为如下几个章节:

\textbf{Linux基础}:事实上,在书写各类项目和在GitHub或各种开发者平台冲浪的过程中,你只需要很基础的Linux知识就能慢慢进化成大神,因此,本章力求简约,意在总结Linux的基础操作,介绍Linux的基本特性,让初学者以此为基础, 探索属于自己的Linux编程之路。

\textbf{序列比对和数据库介绍}:序列比对和数据库使用是课程的重点,作为后续课程的前置知识,我们在此处简单向大家介绍序列比对原理和常见数据库使用方法介绍。

\textbf{EMBOSS、BLAST和HMMER}:作为生物信息的重点和Linux实践,我们将分三章,从易到难介绍当前常用的三种序列比对方法。当然,由于EMBOSS是一个软件包,除序列比对外它还有许多功能,不过其他功能不是我们的介绍重点,留给大家自由探索。

\textbf{ChIP-seq和RNA-seq}:我们通过ChIP-seq和RNA-seq两个项目向大家介绍测序数据上下游处理的全流程,这既是一个新的主题,也是此前所学内容的综合,你可以尝试用shell编程等方式将所学知识融会贯通。

\textbf{TBtools}:作为“番外篇”,我们在此介绍由个人开发者开发的生物信息整合工具——TBtools。

\vspace{0.5cm}

除了Linux和生物信息的主线,我们特将以下主题放置于附录中,此举主要出于便于大家随用随查的考量,并不意味着它们不重要:

\textbf{环境配置}:你可以在这里看到Linux环境配置,conda环境管理器,docker容器技术和vsCode代码编辑器的介绍。你当然可以选择在此处将日后可能需要的环境一并配好,提供便利;不过这部分内容设置的本意是作为“字典”,随用随查,不一定需要从头看到尾,在学习和实践的过程中,你或许会找到你最喜欢的环境配置方法。

\textbf{Git和GitHub}:Git是一个版本控制神器,让我们摆脱手动进行版本控制的烦恼,同时方便与其他人的合作,本身已是一个足够好用的工具,基于Git的GitHub网站则是将这个工具用到了极致。作为开发者,个人开发的能力固然重要,但了解已有的开发成果、快速学习使用,加入社群与其他优秀的开发者建立合作关系才是人类生为群体动物最大的优势。GitHub是世界上最大的开发者社群,它常常能给你很多惊喜!

\begin{center}
    \textit{“君子生非异也,善假于物也。”    ——《荀子·劝学》}
\end{center}

最后,小组的GitHub仓库\href{https://github.com/Alchemiiiist/LEB23\_G4}{LEB23\_G4}记录了我们所有的报告和大部分代码,本期末报告的Latex源代码也在仓库中。预祝大家学习愉快!也希望能和大家一起不断完善此文档,和罗老师一起将《Linux生物信息技术基础》这门课越办越好!
