\chapter{Linux及操作系统简介}

Linux是一种免费的开源操作系统。它基于Unix操作系统,由Linus Torvalds于1991年创建。Linux用于许多不同的设备,例如服务器,超级计算机,智能手机甚至汽车。它以其稳定性,安全性和灵活性而闻名。Linux有许多不同的发行版或“distros”,例如Ubuntu,Fedora,CentOS,Gentoo,Arch Linux等。

Linux是基于命令行的操作系统,这意味着用户使用文本命令与之交互,而不是像Windows或Mac OS X那样使用图形用户界面(GUI)。它使用户对其系统拥有更多的控制权,并使他们更容易自动化任务。

\section{Linux发行版}
Linux发行版是一个由Linux内核、GNU工具、附加软件和软件包管理器组成的操作系统。它也可能包括显示服务器和桌面环境,以用作常规的桌面操作系统。这个术语之所以是“Linux发行版”,是因为像Debian、Ubuntu这样的机构“发行”了Linux内核以及所有必要的软件及实用程序(如网络管理器、软件包管理器、桌面环境等),使其可以作为一个操作系统使用。

一些流行的Linux发行版包括Ubuntu,Fedora,CentOS,Gentoo,Arch Linux等。这些发行版都有自己的特点和优点,列举如下:
\begin{itemize}
    \item Fedora:面向开发人员,新版本的软件包更新频繁。
    \item Debian:稳定,易于使用,适合服务器。
    \item Ubuntu:易于使用,适合桌面用户。
    \item CentOS:企业级操作系统,稳定性高。
    \item Gentoo:高度可定制,适合高级用户。
    \item Arch Linux:轻量级和自定义。
\end{itemize}

在本学期中,我们使用的是Ubuntu。
