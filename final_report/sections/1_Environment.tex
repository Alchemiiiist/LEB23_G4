\chapter{环境配置}

\section{Linux环境配置}
Windows和MacOS是当前最流行的两个操作系统,其中,MacOS是类Unix系统,与Linux共享编程语言,因此你可以在Terminal中直接使用Linux相关指令。如果你是Windows系统,你需要安装Linux子系统或虚拟机。

\subsection{Windows Subsystem for Linux, WSL}
WSL是Windows Subsystem for Linux的缩写,是一个在Windows 10上能够运行原生Linux二进制可执行文件(ELF格式)的兼容层。它是由微软与Canonical公司合作开发,其目标是使纯正的Ubuntu、Debian等映像能下载和解压到用户的本地计算机,并且映像内的工具和实用工具能在此子系统上原生运行。

WSL提供了一个完整的Linux内核,但它不是一个虚拟机。相反,WSL提供了一个Linux系统调用兼容层,以便可以在Windows上运行原生Linux二进制文件。这意味着您可以在Windows上运行Linux命令行工具、脚本和应用程序,而无需使用虚拟机或双引导设置。

\begin{quotation}

\end{quotation}

\subsection{虚拟机}

\section{conda:管理包}

\section{vscode:集成开发平台}

